\section{Loop Recognition:  Big Picture}

Loop closure detection is a tecnique that helps to reduce the escalating
uncertainty that is generated from the iterative extimation of the evolution
of the system state.\\ In particular it can be applied
in autonomous vehicle navigation tasks where the enviromental informations
gathered from the sensors can be used to build an artificial map.\\
The main objective carried out from loop closure algorithms is to identify
portions of the environment that have strong correspondence with others
already seen and then decide if the robot is navigating in a location of
the environment in which it has already  been.\\
In order to develop a loop closure algorithm, that could work in a
simulation, it is necessary to have a dataset and other foundamental components:
\begin{itemize}
  \item A feature extractor able to generate consistent features described
    by possibly unique descriptors.
   \item A data association euristic that allows to find which one of the
     new features corresponds to the older.
   \item An extimator of the evolution of the state, that given the data
     association, iteratively minimise the error of the guess.\\
     Since in this application it deals with laser scan measurements
     it would be called scan-matcher.
     \item An algorithm for loop closures recognition that processes the
       data memorised and tries to close a loop on the artificial map.
\end{itemize}

\section{ C++ Implementation Details}
\subsection{ Dataset Parser}
I was given a dataset of a real 2D laser scanner featuring an autonomous
mobile robot navigating through a corridor.\\ The DatasetManager
class gathers all the scanner ranges in data structures according
to their sequence number.
\subsection{ Feature Extractor}
The feature of choice is the line, precisely a 2D segment.\\
The class LineMatcher, through its euristic, tries to align subsequent
cartesian points that corresponds to the spots where the laser hits the
walls.\\ When the distance or the orientation between two subsequent points
exceedes a threshold a segment is built with the former and previously
encountered points, then the latter will be the beginning of the following
segment.
\subsection{ Data Associator}
In order to recognise features among two consecutive sets of laser ranges
that corresponds to the same environment entities, the class DataAssociator
orders the possible correspondences assigning a score based on a similarity
measure.\\This measure has been choosen  to be almost invariant to the
rototraslation  of the robot that happens between two scan matches.\\
In fact the length of the segments and  also the orientation, assuming a
that the interval between scans is very small, would give assurance of
only small changes.\\ Nevertheless the final decision for the best
association is influenced, where it is possible to compute it, by an
additional metric, which is the similarity between the sum of the angles
that span from the examined segments and their respective neighboring
segments.\\ In fact this quantity would be invariant also with the respect
of a significant rototranslation since it is an absolute quantity, not
dependant on the robot orientation.

\subsection{ Scan Matcher }
This class is responsible for the extimation of the rototranslation
between scan matches.\\Once that some data associations has been
identified, an iterative optimization (ICP) would be applied on
points of the associated segments to recover the homogeneous
transformation expressed in terms of SE(2) with a matrix
$T(\Delta x,\Delta y,\Delta \theta)$.\\
The state evolution is carried out on a manifold and stored on a graph.

\subsection{ Graph Manager}
In the nodes of the graph is contained the state extimated at each
iteration togheter with the features extracted and the laser scanner ranges,
Moreover in the edges there are the transformations coming from scan-matching
and also the data associations found.\\
With all this informations the Graph class carries out a preprocessing
operation in order to reduce the complexity of the loop closure phase:
it tries to assemble the features of subsequent iterations in \textsl{trails}.\\
It means that if a certain segment of scan $k_n$ is associated with another 
that belongs to the scan $k_{n+1}$ and finally the second is associated with a third
belonging to scan $k_{n+2}$ they will be assembled in a \textsl{trail}.\\
A \textsl{trail} can be treated as a single feature that could be identified
with the first segment of the trail, addictionally the more would be is length,
the more \textsl{weigth} it will have on the loop recognition procedure.

\subsection{ Loop Recognitor}
The LoopCloser class uses a distance map approach to evaluate loop closure
candidates.\\At first, whenever the loop recognition function is triggered,
it divide all the available \textsl{trails} at the current iteration in two
subsets: the \textbf{query set}, which contains the most recent ones, and the 
\textbf{tree set} in which there are all the ones that belong to past 
iterations ( in particular only to iterations that are remote in time ).\\
Then it exploits the KD tree implementation available and feed it with
the \textbf{tree set} \textsl{trails}. Then it queries the KD tree with the
other set to obtain some candidates.\\
After the aforementioned step, those candidates will be validated and then a score will be 
assigned to every one of them.\\Only the candidates that have a score greater
than a threshold will be treated as legitimate loop closures.




\subsection{ GUI rendering and tests}
At last, all the graphical inteface animation has been created
using the opencv library.\\Consequently all the computations to generate the
the points to be colored are being carried out from the Map and Drawer classes.\\
The overall procedings of the project has been verified with unit tests
to certificate function behaviours, avoiding the introduction of regression
bugs.\\The eigen library has been used extensively to operate matrix arithmetics.

\section{ Conclusions and afterthoughts}
In conclusion, it can be said that the algorithm recognises some loops, and 
there are several hundreds of iterations between the recognised loop ends.
This is extremely encouraging since it is dealing with a real dataset.\\
The complexity grows more than linearly with further iterations.\\ This means
that an efficient solution would be to run the loop recognition portion in
parallel, for example on another threads, with a shared data structure that
would be the graph.







